\section{HARDWARE PLATFORM}
\subsection{Structure of Link Module}
We constructed the link module of the multilinks as shown in \figref{hardware}(a). Each link module has a build-in propeller at the center and a servo motor at the end of the link module comprised the main part of the joint module. The rotation range of each joint is $-\frac{\pi}{2}$[rad] $\sim \frac{\pi}{2}$[rad]. The multilinks can transform only in two-dimensions due to the joints with the same rotational axis. The range of lifting force is (0[N]$\sim$16.5[N]). The length of each link is 0.6[m] while the diameter of the propeller protector is 0.38[m]. 

\begin{figure}[t]
  \begin{center}
    \includegraphics[width=1.0\columnwidth]{figs/hardware.pdf}
  \end{center}
  \caption{The hardware components of the multilinked multirotor. (a): the link module of multilinks with built-in propeller at the center, servo motor at the end and controller board(Neuron). (b): controller board Neuron communicating with each device and comprising IMU onboard. (c): controller board Spinal communicating with each Neuron and the onboard computer. (d): transformable hex-rotor with six-links consisting of the link modules. \label{figure:hardware}}
\end{figure}

\subsection{Multi-Layer Structure for Internal Communication}
The multirotor with multilinks has not only motors for rotation of propeller but also servo motors for transform as actuator. Besides, the length of the multilinks is longer than conventional aerial robots. In the conventional method, an aerial robot has only one central processor connected to all actuators and sensors. However, it is assumed that if we provide the conventional method for the multilinks, the possibility of disconnection increases due to the length of the multilinks. Matsui et al.\cite{Matsui2005} made reference to this problem for humanoid robots.
\par
For the reason noted above, we constructed multi-layer structure for internal communication(\figref{internal_communication}). As shown in \figref{hardware}(a), each link module has the controller board(Neuron)(\figref{hardware}(b)) which sends data to the motor and the servo motor and receives data from IMU(Inertia Measurement Unit). Moreover, the center link has the other controller board(Spinal)(\figref{hardware}(c)) which communicates with each Neuron. We use CAN(Controller Area Network)\cite{CAN} to construct the communication system between the controller boards. As used in cars, CAN is a reliable communication system which resists external noise. Thereby, the amount of electric wiring decreases resulting that the reliability of the communication improves. 
\par
Moreover, the computer is connected to Spinal on USB-serial communication. The LQI control, motion planning, etc are executed on the computer.

\begin{figure}[t]
  \begin{center}
    \includegraphics[width=1.0\columnwidth]{figs/internal_communication.pdf}
  \end{center}
  \caption{The multi-layer structure for internal communication comprising Neuron, Spinal and Computer: Neuron communicates with each device(motor, servo motor and IMU) while Spinal communicates with Neuron by using CAN. Computer and Spinal communicate by using USB-Serial.\label{figure:internal_communication}}
\end{figure}
